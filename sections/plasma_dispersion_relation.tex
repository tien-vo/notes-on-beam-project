\section{Kinetic plasma dispersion relation}\label{sec:dispersion relation}

First, we establish the conditions for the formulation of waves in a kinetic
plasma. Specifically, the focus is on the instability in beam-plasma processes
and beam evolution. In this section, we will heavily rely on Benz textbook for
the theoretical discussion in \cite{Benz1993}. Under a classical, macroscopic
electromagnetic field, the Vlasov equation describes the dynamics of one
particle species $f_s$ in configuration space $(\vb{r}_s,\vb{p}_s=m\vb{v}_s)$ is

\begin{equation}\label{eq:vlasov equation}
    \frac{\partial f_s}{\partial t}
    +\vb{v}_s\vdot\grad{f_s}
    +\frac{q_s}{m_s}\qty(\vb{E}+\vb{v}_s\times\vb{B})
    \vdot\frac{\partial f_s}{\partial\vb{v}_s}
\end{equation}
where we have assumed a collisionless distribution $f_s$ of particles. Now, we
suppose the system is linearized with a homogenous and stationary unperturbed
distribution $f_s^0$ and write
$f_s\qty(\vb{r}_s,\vb{v}_s,t)
=f_s^0\qty(\vb{v}_s)+f_s^1\qty(\vb{r}_s,\vb{v}_s,t)$.
Suppose electric oscillations are aligned along the background magnetic field
$(\vb{k}\parallel\vb{B}_0)$. Then \cref{eq:vlasov equation} under a Fourier
transform (such that $\grad\to i\vb{k}$ and $\partial_t\to-i\omega$) is
\begin{equation}
    -i\omega f_s^1
    +i\vb{k}\vdot\vb{v}_sf_s^1
    -i\frac{q_s\phi}{m_s}\vb{k}\vdot\frac{\partial f_s^0}{\partial\vb{v}_s}=0
\end{equation}
where $\phi(\vb{r}_s,t)$ is the scalar potential such that $\vb{E}=-\grad\phi$.
We can then invert and solve for $f_s^1$
\begin{equation}\label{eq:distribution perturbation}
    f_s^1
    =\frac{q_s\phi}{m_s}
    \frac{\vb{k}\vdot\partial f_s^0/\partial\vb{v}_s}
    {\vb{k}\vdot\vb{v}_s-\omega}
\end{equation}
which normalizes to the density $n_s$. From Poisson's equation
$k^2=4\pi\sum_sq_sn_s$, we can write the general dispersion relation of a
kinetic plasma
\begin{equation}\label{eq:general dispersion relation}
    k^2=\sum_s\frac{4\pi q_s^2}{m_s}
    \int_{\R^3}d\vb{v}_s\frac{\vb{k}\vdot\partial f_s^0/\partial\vb{v}_s}
        {\vb{k}\vdot\vb{v}_s-\omega}
\end{equation}
This corresponds directly with $H\qty(k,\omega/k)$ from (5.2.16) in
\cite{Benz1993}. Under an extension to the complex field, suppose
$\omega=\Omega_k+i\gamma_k$ and the integral can be evaluated in the curve
$C=\R^3\cap\C^3$. Since it is only of physical interest that
$\partial f_s^0/\partial\vb{v}_s$ is analytic over $\C^3$, the integrand in
\cref{eq:general dispersion relation} is meromorphic with a pole at
$\vb{k}\vdot\vb{v}_s=\omega$. From Sokhotski-Plemeji theorem, the integral
evaluates as
\begin{equation}\label{eq:3D integral evaluation}
    \int_Cd\vb{v}_s
        \frac{\vb{k}\vdot\partial f_s^0/\partial\vb{v}_s}
        {\vb{k}\vdot\vb{v}_s-\omega}
        =\pval\int_Cd\vb{v}_s\frac{\vb{k}\vdot\partial f_s^0/\partial\vb{v}_s}
            {\vb{k}\vdot\vb{v}_s-\omega}
            +i\pi\Eval{\frac{\partial f_s^0}{\partial\vb{v}_s}}
                {\vb{k}\vdot\vb{v}_s=\omega}{}
\end{equation}
where $\pval$ is the Cauchy principal value and we have assumed $k>0$.

Now, we consider the formulation of wave modes from the general plasma
dispersion relation in \cref{eq:general dispersion relation} applied to
Maxwellian particle distributions. For simplicity, let us also assume a
one-dimensional system where $C=\R\cap\C$ and
\begin{equation}\label{eq:maxwellian distribution}
    f_s^0=\frac{n_s}{\sqrt\pi v_{th,s}}e^{-\eta_s^2}
\end{equation}
where $\eta_s=(v_s-u_s)/v_{th,s}$, $u_s$ is the drift speed, and
$v_{th,s}=\sqrt{2T_s/m_s}$ is the thermal speed. Also, let
$\lambda_s=(\omega/k-u_s)/v_{th,s}$. Then the dispersion relation now reads
$k^2=\sum_sk_s^2$ where
\begin{equation}\label{eq:dispersion relation contributions}
    k_s^2=\frac{4\pi q_s^2}{m_sv_{th,s}}
        \int_Cd\eta_s\frac{\partial f_s^0/\partial\eta_s}{\eta_s-\lambda_s}
\end{equation}
Similar to \cref{eq:3D integral evaluation}, the integral evaluates as
\begin{equation}
    \int_Cd\eta_s\frac{\partial f_s^0/\partial\eta_s}{\eta_s-\lambda_s}
    =\pval\int_Cd\eta_s\frac{\partial f_s^0/\partial\eta_s}{\eta_s-\lambda_s}
    +i\pi\Eval{\frac{\partial f_s^0}{\partial\eta_s}}{\eta_s=\lambda_s}{}
\end{equation}
Since $f_s^0$ is known, the principal part is
\begin{equation}\label{eq:1D_integral_evaluation}
    \pval\int_Cd\eta_s\frac{\partial f_s^0/\partial\eta_s}{\eta_s-\lambda_s}
    =-\frac{2n_s}{\sqrt\pi v_{th,s}}
        \pval\int_Cd\eta_s\frac{\eta_s}{\eta_s-\lambda_s}e^{-\eta_s^2}
\end{equation}

Note that we have assumed $\omega=\Omega_k+i\gamma_k$ where
$\Omega_k,\gamma_k\in\R$. The electric field, for example, can be written as
\begin{equation}
    E(z,t)=E_0e^{i(kz-\omega t)}=E_0e^{i(kz-\Omega_kt)}e^{-\gamma_kt}
\end{equation}
The real frequency $\Omega_k$ contributes to the phase velocity
$v_\phi=\Omega_k/k$ of the generated waves. The imaginary term $\gamma_k$
describes either the damping or growth of the field magnitude depending on its
sign. This mechanism is called Landau damping. From the form of \cref{eq:3D
integral evaluation}, it is apparent that this imaginary term comes from the
contribution of the pole at the resonant phase velocity $v_\phi$. Thus, it is
closely connected to particles in resonance with the generated waves. In the
wave frame, these particles are able to effectively exchange energy with
the wave, either making it grow or damp away exponentially.

