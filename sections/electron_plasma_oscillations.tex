\section{Plasma waves}\label{sec:plasma waves}


In this section, we consider some wave modes permitted by the kinetic plasma
dispersion relation and their growth rate in different regimes.
The background population usually consists of electrons and ions. Electrons,
being the less inertial particles (since $m_e/m_i\sim1/1836$), are more mobile.
They can lose and gain energy to and from the waves much more easily than the
ions. Thus, the most dominant plasma wave mode is the electron oscillations and
we can consider background ions as stationary.

\subsection{Background electrons}

Now, background electrons have $u_e=0$. So $\lambda_e=\omega/kv_{th,e}$.
Generally, the electron beam of interest has $u_{be}\gg v_{th,e}$. Thus, we also
assume $\abs{\lambda_e}\gg1$. Under a Taylor expansion, the integral in
\cref{eq:1D_integral_evaluation} becomes

\begin{equation}
    \pval\int_Cd\eta_e\frac{\eta_e}{\eta_e-\lambda_e}e^{-\eta_e^2}
    =-\frac1{\lambda_e}\pval\int_Cd\eta_ee^{-\eta_e^2}
        \sum_{j=0}^\infty\frac{\eta_e^{j+1}}{\lambda_e^j}
    \approx-\frac{\sqrt\pi}{2}\qty(\frac1{\lambda_e^2}
        +\frac32\frac1{\lambda_e^4})
\end{equation}
The contribution from the pole at $\eta_e=\lambda_e$ scales as
$e^{-\lambda_e^2}$. Thus, in terms of magnitude, it is negligible and we can
write the contribution of the background electrons to the dispersion relation as
\begin{equation}
    k_e^2=\frac{k^2\omega_{pe}^2}{\omega^2}
        \qty(1+\frac32\frac{k^2v_{th,e}^2}{\omega^2})
\end{equation}
Neglecting the other species' contributions, we can invert the expression and
write
\begin{equation}
    \omega^2=\omega_{pe}^2\qty(1+\frac32\frac{k^2v_{th,e}^2}{\omega^2})
\end{equation}
The first term is the solution for a cold plasma, which results in
$\omega=\omega_{pe}$. The second term gives the next non-trivial correction for
$T_e\neq0$ which, to the first order, is the Bohm-Gross dispersion relation
\begin{equation}
    \omega^2=\omega_{pe}^2+\frac32k^2v_{th,e}^2
\end{equation}

Now, to approximate the imaginary term, we use the cold plasma solution
$\omega=\omega_{pe}$ and write the electron contribution to $k^2$ in full
form
\begin{equation}
    k_e^2=\frac{k^2\omega_{pe}^2}{\omega^2}
    \qty(1
        +\frac{i\pi}{n_ev_{th,e}}\frac{\omega^2}{k^2}
        \Eval{\frac{\partial f_e^0}{\partial\eta_e}}{\eta_e=\lambda_e}{})
\end{equation}
Again, ignoring the other contributions, we can invert and find
\begin{equation}
    \omega=\qty(\omega_{pe}^{-2}
    -\frac{i\pi}{k^2n_ev_{th,e}}
    \Eval{\frac{\partial f_e^0}{\partial\eta_e}}{\eta_e=\lambda_e}{}
    )^{-1/2}=\qty(\omega_{pe}^{-2}-i\epsilon)^{-1/2}
\end{equation}
As established above, $\epsilon$ is small. So a Taylor expansion brings the
solution to the form
\begin{equation}
    \omega\sim\omega_{pe}\qty(
        1+\frac{i\pi}{2}\frac{\omega_{pe}^2}{k^2n_ev_{th,e}}
        \Eval{\frac{\partial f_e^0}{\partial\eta_e}}{\eta_e=\lambda_e}{}
    )
\end{equation}
Then it follows that
\begin{equation}
    \gamma_k=-\sqrt\pi\frac1{\qty(k\lambda_D)^3}\exp\qty[-\frac1{(k\lambda_D)^2}]<0
\end{equation}
where $\lambda_D=v_{th,e}/\omega_{pe}$ is the Debye length. The growth rate
is always negative, which means this wave mode is damped everywhere. This is
only valid for our assumption regarding $\lambda_e$, which is when
$k\lambda_D\ll1$. For $k\lambda_D\geq0.5$, the rate grows linearly before it
decreases exponentially and our assumption of small $\epsilon$ is violated.

